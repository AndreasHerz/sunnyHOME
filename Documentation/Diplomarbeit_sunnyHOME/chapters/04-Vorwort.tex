% !TEX root = ../Vorlage_DA.tex
%	########################################################
% 							Vorwort
%	########################################################


%	--------------------------------------------------------
% 	Überschrift, Inhaltsverzeichnis
%	--------------------------------------------------------
\chapter*{Vorwort}
\addcontentsline{toc}{chapter}{Vorwort}


%	--------------------------------------------------------
% 	Inhalt
%	--------------------------------------------------------
Sehr geehrte Damen und Herren, liebe Leserschaft, \newline


vor Ihnen findet sich eine Diplomarbeit zum Thema „sunnyHOME – Umweltdatenerfassung“, welche von dem HTL-Braunau Schüler Andreas Herz verfasst wurde. Dieser besuchte an der HTL Braunau den schulautonomen Zweig „Communications“, und schieb auf Grund des Abschlusses die vor Ihnen liegende Diplomarbeit.\newline

Ende des letzten Schuljahres (2017/18), entschloss ich mich (Andreas Herz), mit einem Klassenkameraden, die Themenstellung einer Energie-autarken Wetterstation in Angriff zu nehmen. Da dieser Klassenkamerad jedoch kurzfristig nicht mehr aufsteigen durfte, wurde das Projekt zu einer Einzelarbeit.\newline

Mit Hilfe von DI. Roland Sageder, welcher der zugewiesene Betreuer des Projektes war, konnten wir uns bereits vor den Sommerferien ein grobes Bild des Vorhabens machen. Auch wenn in den Sommerferien 2018 recherchiert wurde, mussten nach der vorgezogenen Matura die Pläne, auf Grund der plötzlichen Einzelarbeit, verändert werden. Das Projekt musste auf die Grundfunktionen begrenzt werden, und Zeitpläne wurden erstellt. Frau Margit Fuchs, welche zur Aushilfe beim Projektmanagement zugeteilt war, half mir mit guten Vorschlägen dabei. Während des Jahres wurde aber immer wieder klar, dass man für Softwareprojekte oft mehr Zeit braucht als angesetzt ist. Immer wieder gab es Hindernisse und Pannen, in welchen ich mir zusätzliches Wissen aneignen musste um weiterzukommen. Im Endeffekt sehe ich dann jedoch ein, dass jede Panne eine lehrreiche Lektion war.\newline

Hiermit möchte ich die Gelegenheit nutzen, um meinen Dank an all diejenigen auszusprechen, die mir bei meinem Projekt geholfen haben. Insbesondere möchte ich dem Herrn Dipl. Ing. Roland Sageder danken. Für seine Geduld mit mir, in Situationen in denen ich vielleicht etwas mehr als nur einmal eine Erklärung gebraucht habe und auch die Motivation, welche er mir immer wieder gegeben hat. Aber auch Frau Margit Fuchs möchte ich danken. Ohne sie hätte ich das Projekt definitiv nicht so gut organisieren können. \newline

Ich wünsche Ihnen nun viel Freunde beim Lesen dieser Abschlussarbeit.\newline

Andreas Herz 
\newline
\newline 4963 St. Peter am Hart, \today
