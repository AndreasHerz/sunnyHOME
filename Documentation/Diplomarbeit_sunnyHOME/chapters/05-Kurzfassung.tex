% !TEX root = ../Vorlage_DA.tex
%	########################################################
% 							Zusammenfassung
%	########################################################


%	--------------------------------------------------------
% 	Überschrift, Inhaltsverzeichnis
%	--------------------------------------------------------
\chapter*{Kurzfassung}
\addcontentsline{toc}{chapter}{Kurzfassung}



%	--------------------------------------------------------
% 	Inhalt
%	--------------------------------------------------------
In dieser Diplomarbeit geht es hauptsächlich um die Arduino-Programmierung eines ESP32 im Zusammenhang mit IoT (Internet of Things) und Energy-Harvesting. Für den Endbenutzer soll eine Wetterstation geschaffen werden, welche robust, unabhängig von einer Stromversorgung und leicht zu modifizieren ist. 

„sunnyHOME“ ist ein Gerät, welches Wetterdaten sammelt und kabellos ins Internet publiziert. Es können verschiedene Sensoren, wie zum Beispiel Niederschlags- oder UV-Sensoren, verwendet werden. Die Daten der Sensoren werden dann über einen Mikrocontroller eingelesen und an einen Web-Server gesendet. Dieser wertet die Daten aus, welche dann über eine Website oder eine App visualisiert werden. Somit kann man von überall auf der Welt nachschauen, ob zum Beispiel die Blumen im Garten genug Wasser bekommen oder nicht.   

